\documentclass[a4paper,
               %boxit,        % check whether paper is inside correct margins
               %titlepage,    % separate title page
               %refpage       % separate references
               biblatex,     % biblatex is used
               keeplastbox,   % flushend option: not to un-indent last line in References
               %nospread,     % flushend option: do not fill with whitespace to balance columns
               %hyphens,      % allow \url to hyphenate at "-" (hyphens)
               %xetex,        % use XeLaTeX to process the file
               %luatex,       % use LuaLaTeX to process the file
               ]{jacow}
%
% ONLY FOR \footnote in table/tabular
%
\usepackage{pdfpages,multirow,ragged2e} %
%
% CHANGE SEQUENCE OF GRAPHICS EXTENSION TO BE EMBEDDED
% ----------------------------------------------------
% test for XeTeX where the sequence is by default eps-> pdf, jpg, png, pdf, ...
%    and the JACoW template provides JACpic2v3.eps and JACpic2v3.jpg which
%    might generates errors, therefore PNG and JPG first
%
\makeatletter%
	\ifboolexpr{bool{xetex}}
	 {\renewcommand{\Gin@extensions}{.pdf,%
	                    .png,.jpg,.bmp,.pict,.tif,.psd,.mac,.sga,.tga,.gif,%
	                    .eps,.ps,%
	                    }}{}
\makeatother

% CHECK FOR XeTeX/LuaTeX BEFORE DEFINING AN INPUT ENCODING
% --------------------------------------------------------
%   utf8  is default for XeTeX/LuaTeX
%   utf8  in LaTeX only realises a small portion of codes
%
\ifboolexpr{bool{xetex} or bool{luatex}} % test for XeTeX/LuaTeX
 {}                                      % input encoding is utf8 by default
 {\usepackage[utf8]{inputenc}}           % switch to utf8

\usepackage[USenglish]{babel}

%
% if BibLaTeX is used
%
\ifboolexpr{bool{jacowbiblatex}}%
 {%
  \addbibresource{references.bib}
 }{}
\listfiles

%%
%%   Lengths for the spaces in the title
%%   \setlength\titleblockstartskip{..}  %before title, default 3pt
%%   \setlength\titleblockmiddleskip{..} %between title + author, default 1em
%%   \setlength\titleblockendskip{..}    %afterauthor, default 1em

\begin{document}

\title{Beam optics modelling through fringe fields during injection and extraction at the CERN Proton Synchrotron}

\author{E. P. Johnson\thanks{eliott.philippe.johnson@cern.ch}, M. A. Fraser, M. G. Atanasov, Y. Dutheil, E. Oponowicz,\\ CERN, 1211 Geneva 23, Switzerland}
	
\maketitle

%
\begin{abstract}
As the beam is injected and extracted from the CERN Proton Synchrotron (PS), it passes through the fringing magnetic fields of the main bending units (MUs). In this study, tracking simulations using field maps created from a 3D magnetic model of the MUs are compared to beam-based measurements made through the fast injection and slow extraction regions. The behavior of the fringe field is characterized, and its implementation in the MAD-X model of the machine is described.
\end{abstract}


\section{Introduction}

The CERN PS is composed of 100 combined function main unit magnets.

\subsection{AGS magnets}
The PS is a Alternating Gradient Synchrotron (AGS) that uses a strong focusing mechanism. It uses combined function magnet that have an half focusing Fig.\ref{fig:focusing} and half defocussing Fig.\ref{fig:defocusing} part of opposite gradient of induction $\frac{dB}{dr}$.

\begin{figure}[!htb]
  \centering
  \begin{minipage}[b]{0.45\columnwidth}
    \includegraphics*[width=\textwidth]{focusing}
    \caption{Closed Focusing yoke}
    \label{fig:focusing}
  \end{minipage}
  \hfill
  \begin{minipage}[b]{0.45\columnwidth}
    \includegraphics*[width=\textwidth]{focusing}
    \caption{Open Defocusing yoke}
    \label{fig:defocusing}
  \end{minipage}
\end{figure}

The main units of the PS are combined function magnet. The yoke provides both a dipolar and a quadrupolar field. Each magnet units consists in a combination of five focusing yokes followed by five defocusing yokes following the FDODFOFDODF pattern \cite{}

There are four types of magnets, U, R, S, T that have a different combination of focusing and defocusing half-magnet and the coil on the inside or outside of the ring.

Although the combined function magnet is a clever design when the beam is in orbit, extracting the beam through the main units causes stray-field effects.

\section{OPERA model}
A model of the PS magnet was developed \cite{anglada_reference_nodate}. It allows producing field maps Fig.\ref{fig:dipole_field} and Fig.\ref{fig:gradient_field}. Limitation are that it is straight.
The OPERA model gives similar results to the measurement done in 1977 \cite{manglunki_beam_1997} but with increased flexibility as one can produce field maps at different energies and the field map can have a greater distribution in space as there is no limitation by the size of the measuring equipment.

\begin{figure}[!htb]
   \centering
   %trim={<left> <lower> <right> <upper>}
   \includegraphics*[width=1.0\columnwidth, trim={0 2.9cm 0 4cm},clip]{dipole_field.png}
   \caption{Dipole field map of a U-type magnet centered in the vertical plane at 24 GeV}
   \label{fig:dipole_field}
\end{figure}

\begin{figure}[!htb]
   \centering
   %trim={<left> <lower> <right> <upper>}
   \includegraphics*[width=1.0\columnwidth, trim={0 2.0cm 0 4cm},clip]{gradient_field.png}
   \caption{Gradient field of a U-type magnet centered in the vertical plane at 24 GeV}
   \label{fig:gradient_field}
\end{figure}

Tracking a particles through these field maps is done through a script by Ewa and Yann. It also allows to produce transfer matrices that can be implemented in tracking simulation such as MAD-X.

\section{Injection BTP}

As the beam is injected via the BTP transfer line to the PS ring, it passes through the stray fields of the PR.BHT41 T-type main unit magnet. Once through, the beam is kicked by the PI.SMH42 septum magnet to capture the reference orbit. Beam position can then be measured using a Secondary Emission Grid (SEM Grid) PI.BSG42.

\begin{figure}[!htb]
   \centering
   \includegraphics*[width=1.0\columnwidth]{kick_response_old_model.png}
   \caption{Kick response as a function of POPS}
   \label{fig:kick_response_old_model}
\end{figure}

\section{Extraction}

The extraction to the east area passes through two main units, MU62 and MU63. As a result of the stray fields, instead of following a DFOFDODFOF lattice, the beam follows a DFOFDODD?D lattice. This succession in defocusing gradient increases the horizontal beam size. This can be seen in the trajectory that the beam takes as the beam is extracted from the PS to the East Area Fig. \ref{fig:gradient track mu62}. The beam is defocused in the half-defocusing part of the magnet but once it enters the second half-focusing part it is transversely out of the flat zone of the gradient and into the non-linear part that is has an order of magnitude larger defocusing effect. This leads to a severe increase in horizontal beam size and thus a very flat beam at the exit of the MU62. This is probably why the Q74 magnet was installed and is allowed to pulse so high.

\begin{figure*}[!htb]
   \centering
   \includegraphics*[width=\textwidth]{track_mu62.png}
   \caption{track mu62}
   \label{fig:track mu62}
\end{figure*}

\begin{figure}[!htb]
   \centering
   \includegraphics*[width=1.0\columnwidth]{gradient_track_mu62.png}
   \caption{gradient track mu62}
   \label{fig:gradient track mu62}
\end{figure}


\section{Magnetic shims}

Magnetic shims are installed at ejection 16 and 63 to homogenize the field by shielding the ejected beam from a non linear fringe field. In 16, the shims have different radial position for each of the five different shims.

Look at this shims in Fig. \ref{fig:shim}.

\begin{figure*}[!tbh]
    \centering
    \includegraphics*[width=\textwidth]{shim}
    \caption{Shim in the ejection 16}
    \label{fig:shim}
\end{figure*}


\section{CONCLUSION}

In conclusion,...

% REFERENCES
\printbibliography



\end{document}