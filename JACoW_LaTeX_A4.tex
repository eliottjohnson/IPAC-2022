\documentclass[a4paper,
               %boxit,        % check whether paper is inside correct margins
               %titlepage,    % separate title page
               %refpage       % separate references
               biblatex,     % biblatex is used
               keeplastbox,   % flushend option: not to un-indent last line in References
               %nospread,     % flushend option: do not fill with whitespace to balance columns
               %hyphens,      % allow \url to hyphenate at "-" (hyphens)
               %xetex,        % use XeLaTeX to process the file
               %luatex,       % use LuaLaTeX to process the file
               ]{jacow}
%
% ONLY FOR \footnote in table/tabular
%
\usepackage{pdfpages,multirow,ragged2e} %
%
% CHANGE SEQUENCE OF GRAPHICS EXTENSION TO BE EMBEDDED
% ----------------------------------------------------
% test for XeTeX where the sequence is by default eps-> pdf, jpg, png, pdf, ...
%    and the JACoW template provides JACpic2v3.eps and JACpic2v3.jpg which
%    might generates errors, therefore PNG and JPG first
%
\makeatletter%
	\ifboolexpr{bool{xetex}}
	 {\renewcommand{\Gin@extensions}{.pdf,%
	                    .png,.jpg,.bmp,.pict,.tif,.psd,.mac,.sga,.tga,.gif,%
	                    .eps,.ps,%
	                    }}{}
\makeatother

% CHECK FOR XeTeX/LuaTeX BEFORE DEFINING AN INPUT ENCODING
% --------------------------------------------------------
%   utf8  is default for XeTeX/LuaTeX
%   utf8  in LaTeX only realises a small portion of codes
%
\ifboolexpr{bool{xetex} or bool{luatex}} % test for XeTeX/LuaTeX
 {}                                      % input encoding is utf8 by default
 {\usepackage[utf8]{inputenc}}           % switch to utf8

\usepackage[USenglish]{babel}

%
% if BibLaTeX is used
%
\ifboolexpr{bool{jacowbiblatex}}%
 {%
  \addbibresource{references.bib}
 }{}
\listfiles

%%
%%   Lengths for the spaces in the title
%%   \setlength\titleblockstartskip{..}  %before title, default 3pt
%%   \setlength\titleblockmiddleskip{..} %between title + author, default 1em
%%   \setlength\titleblockendskip{..}    %afterauthor, default 1em

\begin{document}

\title{Beam optics modelling through fringe fields during injection and extraction at the CERN Proton Synchrotron}

\author{E. P. Johnson\thanks{eliott.philippe.johnson@cern.ch}, M. A. Fraser, M. G. Atanasov, Y. Dutheil, E. Oponowicz,\\ CERN, 1211 Geneva 23, Switzerland}
	
\maketitle

%
\begin{abstract}
As the beam is injected and extracted from the CERN Proton Synchrotron (PS), it passes through the fringing magnetic fields of the main bending units (MUs). In this study, tracking simulations using field maps created from a 3D magnetic model of the MUs are compared to beam-based measurements made through the fast injection and slow extraction regions. The behavior of the fringe field is characterized, and its implementation in the MAD-X model of the machine is described.
\end{abstract}


\section{Introduction}
As protons and ions beams are injected and extracted in the PS ring, they travel through non-linear stray fields produced by the PS MUs. In these regions, an accurate optic model is imperative for the preservation of transverse emittance. Scaling of the model with energy is required as the injection takes place at 1 GeV/c, and extraction to the East Area and to the Super Proton Sychrotron (SPS) are at 24 GeV/c and 26 GeV/c respectively. Limitation of earlier models based on measurements were that they were produced at a single energy. Versatility is also considered as the beam travels through stray fields with different angles, different magnet types, and a non-constant number of MUs. Previous models would only consider a single track to produce the relevant optical parameters from the measurements. The model would be useful to describe slow extracted beams for Charm High-energy Ions for Micro Electronics Reliability Assurance (CHIMERA), which aims to use very high-energy (VHE) heavy-ion beams to study the effect of single event effects (SEE) on electric, electronic, and electromechanical (EEE) devices, both for research and industry users. A new technique for modeling the stray field that relies solely on simulations is presented, considering the proton injection at 2 GeV/c and the extraction a 24 GeV/c to the East Area.

\section{Field maps}
\subsection{PS Main Units}
The CERN PS uses the strong focusing mechanism for beam convergence. The PS is composed of 100 combined function main unit magnets that produce dipolar and quadrupolar fields simultaneously. Each magnet is divided into two half-units with gradients of opposite polarity. Half-units are composed of five blocks of either closed blocks that are focusing (F), see Fig.\ref{fig:focusing} or open blocks that are defocusing (D), see Fig.\ref{fig:defocusing}.

\begin{figure}[!htb]
  \centering
  \begin{minipage}[b]{0.45\columnwidth}
    \includegraphics*[width=\textwidth]{focusing}
    \caption{Closed Focusing yoke.}
    \label{fig:focusing}
  \end{minipage}
  \hfill
  \begin{minipage}[b]{0.45\columnwidth}
    \includegraphics*[width=\textwidth]{defocusing}
    \caption{Open Defocusing yoke.}
    \label{fig:defocusing}
  \end{minipage}
\end{figure}

There are four types of magnets: R, S, T, and U, depending on the arrangement of the half-units (FD or DF) and whether the main coil is on the inside or outside of the ring \cite{steerenberg_fifty_2011}. Additional coils called pole face winding (PFW) and figure-of-eight loop (F8L) are wrapped around the yokes to correct for chromaticity and tune. Although the combined function magnet works well when the beam is in the central orbit, every injection/extraction of the beam travels through the MUs, which causes large stray-field effects. The PS wasn't built with straight sections long enough for injection/extraction at high energy and the deflection angles of the septas are small which forces the ejected beam to travel close to the central orbit and through the stray fields of the MUs \cite{risselada_beam_nodate}.


\subsection{The OPERA model}
A model of the PS main units was developed using Cobham's Opera-3D that allows the production of field maps for all four types of magnet and for different currents in main coil as well as in the PFW and F8L. The model includes the main junction gap of 20 mm (a significant source of fringe fields) between the two half units as well as the mini junctions between open block of 9.75 mm and 7.75 mm between closed blocks. A plane containing the geometry of both open and closed yokes is swept in the longitudinal direction. This allows to keep a high accuracy of the model and reduces the computation time. A limitation of the single plane is that it is a straight model, whereas the real magnets are curved. The density of the mesh is adjusted so that it has a high resolution in close proximity to the central orbit and at the junctions to capture the fringe fields \cite{anglada_reference_2019}.



%\begin{figure}[!htb]
%   \centering
%   %trim={<left> <lower> <right> <upper>}
%   \includegraphics*[width=1.0\columnwidth, trim={0 2.7cm 0 4.1cm},clip]{gradient_field.png}
%   \caption{Gradient field of a U-type magnet centered in the vertical plane at 24 GeV}
%   \label{fig:gradient_field}
%\end{figure}



An example of a field map (vector field of Bx,By,Bz in x,y,z space) produced by the Opera model is presented in Fig.\ref{fig:dipole_field}. The transverse displacement of the peak of the vertical dipole field $B_{y}$ along the z-axis corresponds to the switch from one half units to the next one. The resolution of the field map is high enough to see the mini-junction between the five blocks.

\begin{figure}[!htb]
   \centering
   %trim={<left> <lower> <right> <upper>}
   \includegraphics*[width=1.0\columnwidth, trim={0 2.9cm 0 4.3cm},clip]{dipole_field.png}
   \caption{Dipole field map of a U-type magnet centered in the vertical plane at 24 GeV.}
   \label{fig:dipole_field}
\end{figure}

The gradient can be calculated from the dipole field map using the following formula:
 
$$ \boldsymbol{G}(x_{j},z) = \frac{\Delta\boldsymbol{B}}{\Delta x} = \frac{\boldsymbol{B}(x_{i+1},z) - \boldsymbol{B}(x_{i},z)}{x_{i+1}-x_{i}} $$
where,
$$ x_{j} = \frac{x_{i+1} + x_{i}}{{2}} $$
 
The gradient is constant and alternating in the central orbit region, but reverses signs at the periphery and increases in magnitude in a nonlinear way; see Fig.\ref{fig:gradient_field}. Similar results have previously been made using magnetic measurements of the MU fields \cite{manglunki_beam_1997}. Earlier models of the stray field for extraction 16 at 26 GeV/c proved to be adequate in the description of operational beams. However, the OPERA model adds the possibility to quickly generate field maps at different energies, different PFW, different F8L settings and for different magnet types. Field maps produced with the OPERA model also have virtually infinite lengths, whereas physical measurements are limited by the geometry of the magnet and the measuring instrument (-0.07 $\leq$ x $\leq$ 0.31 m and -2.55 $\leq$ z $\leq$ 2.73 m), \cite{manglunki_beam_1997}.

\begin{figure}[!htb]
   \centering
   \includegraphics*[width=1.0\columnwidth, trim={0 0 0 0cm},clip]{Main_field_B_G.png}
   \caption{Dipole and gradient component of a  PS U-type MU centered in the vertical plane in both half-units at 24 GeV. Outside of the central orbit the gradient is non-linear and at it's peak as an amplitude almost three times higher.}
   \label{fig:gradient_field}
\end{figure}

\subsection{Tracking}
Particle tracking through field maps is done using the Boris algorithm that tracks charged particles in EM-fields using the discretized equation of motion of the Lorentz force\cite{qin_why_2013}\cite{ripperda_comprehensive_2018}. Additionally, numerical tracking of a number of representative particles allows one to produce transfer matrices that can be implemented in tracking simulation such as MAD-X \cite{yoon_method_2013}.

\section{Beam Tracking}
Fields maps have been produced for each magnet type at three different energies: injection at 2 GeV/c with 533 A, slow extraction to the east area at 24 GeV/c with 4642 A, and extraction to the SPS at 26.4 GeV/c with 5386 A. In the following, the injection from the BTP transfer line to the PS and the extraction from the PS to the East Area have been studied.

\subsection{Injection BTP}
As the beam is injected via the BTP transfer line to the PS ring, it passes through the stray fields of the PR.BHT41 T-type MU magnet. The beam traverses mostly through the defocusing half part and feels a non-linear increase in the gradient up to the nominal value in the central orbit. Once through the MU 41, the beam is kicked by the PI.SMH42 septum magnet to capture the reference orbit. The beam position can then be measured using the downstream Secondary Emission Grid (SEM Grid) PI.BSG42. Single particle tracking is done by centering the initial beam in the middle of the injection pipe using the GEODE reference points see Fig. \ref{fig:injection_btp}.

\begin{figure}[!htb]
   \centering
   \includegraphics*[width=1.0\columnwidth]{injection_btp.png}
   \caption{Tracking through MU 41 T-type, momentum 2 GeV/c, extraction offset 54.684 mm, angle 5.093 degrees. It can be noticed that the field map is straight as opposed to the curved technical drawing.}
   \label{fig:injection_btp}
\end{figure}

Measurements have been done by scaling the current in all main units using the Power system for PS main magnets (POPS); for our purpose, we only consider MU41 as it is the only magnet the beam passes through. As expected, an increasing B field shows a decrease in transverse position of the beam as it is bent closer to the inside of the ring on the SEM-grid. The 2 GeV/c T-type field map is scaled by the same factor as the current, and a single particle is tracked through the field map. The position and angle at the end of the exit of the main unit is saved and inserted in the MAD-X PS lattice up to the SEM-Grid location. For comparison, we look at the current implementation of the stray field at injection that consists of a collection of multipoles that were created using past measurements. These are scaled in the same way. Dipole and quadrupole component along the injection track for the multipole model and the tracking model give similar results. Transverse position as a function of the B field at injection matches closely the measurements, though the tracking performs slightly worse than the multipole model, see Fig. \ref{fig:injection_btp_transverse_position}. This can either be because of the mu-metal (high permeability) shielding wrapped around the injection vacuum pipe that isn't accounted for in the field map, the fact that the field map is straight instead of being bent and/or a different initial position and angle of the tracked particle. As expected, no deviation is observed in the vertical plane.

\begin{figure}[!htb]
   \centering
   \includegraphics*[width=1.0\columnwidth]{transverse_position_vs_POPS}
   \caption{Measurements of the BT3 BTP PS kick response as a function of POPS compared with the OPERA tracking model and the multipole field coefficient model.}
   \label{fig:injection_btp_transverse_position}
\end{figure}

In addition, the current model of the stray field composed of 235 multipoles with multipoles K0, K1, K2, and K3 components is compared with the components found with tracking. The dipole component is saved along the track, and the quadrupole component is computed using two adjacent tracks with a slight transverse offset at the intial starting point. The OPERA tracking model gives components close to the previous model. The dipole integral of the OPERA model is $8.36\cdot10^{-3}$ Tm, and the multipole field coefficient (MFC) model gives $8.86\cdot10^{-3}$ Tm, which is a difference of 5.63\% compared to the MFC model. The integral of the quadrupole component of the OPERA model is $-6.68\cdot10^{-2}$ T and the MFC is $-7.43\cdot10^{-2}$ T, a 10.03\% difference compared to the MFC model.


\subsection{Extraction to the East Area}
The extraction to the East Area is affected by the stray fields of two main units, MU62 and MU63, a consequence of the beam having a higher energy, leaving at a shallower angle than at injection. The beam position and angle upstream of a MAD-X simulation are saved and used as a starting position to track a single particle in MU62. The track follows the extraction beam pipe through MU62 and experiences the effect of both half-units. The stray-field effect of the second half-unit is of particular importance, as the beam climbs the gradient hill, it experiences a non-linear gradient with reverse sign. This leads to a succession in defocusing gradients instead of having an alternating gradient that converges the beam. As a result, an increase in the horizontal beam size is expected at the exit of MU62. A distribution of particles created using MAD-X parameters is tracked through MU62 which shows the defocusing effect, see Fig. \ref{fig:particle_distribution}. The strong Q74 quadrupole located at the exit of the main unit helps cancel this adverse effect.

\begin{figure}[!htb]
   \centering
   \includegraphics*[width=1.0\columnwidth]{particle_distribution.png}
   \caption{Particle distribution entering the main unit 62 (red) and exiting the main unit (blue) following the extraction trajectory (no momentum spread in the distribution).}
   \label{fig:particle_distribution}
\end{figure}

\subsection{Magnetic shims}

To counteract stray fields, magnetic shims (low carbon steel shielding) are installed at ejections 16 and 63 to homogenize the field by shielding the ejected beam from the non-linear fringe field \cite{zickler_influence_nodate}. In 16, the shims have different radial position for each of the five different shims whereas in 63, the vacuum pipe is covered with a constant rectangular shim. No shims are installed in Section 62, where the most important stray field effect is expected. Field map produce with the OPERA model do not include the effect of shims and thus only give a partial description of the actual fields. Adding the shims in the OPERA model is considered but requires considerable rework, as the model takes advantage of a straight geometry, whereas the extraction pipe and shims are at an angle with respect to the main units which would increase the computational time.


\section{CONCLUSION}
Measurement at injection allowed us to gain confidence in using the OPERA model in tracking routine to characterize the operational beam and led to the same conclusions as previously made 20 years ago. Next steps are to use the OPERA model to implement transfer maps or multipoles in MADX and check if the initial parameters in the East Area measured empirically are similar to the ones found with tracking through the field maps.

% REFERENCES
\printbibliography

\end{document}