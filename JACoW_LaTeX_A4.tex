\documentclass[a4paper,
               %boxit,        % check whether paper is inside correct margins
               %titlepage,    % separate title page
               %refpage       % separate references
               biblatex,     % biblatex is used
               keeplastbox,   % flushend option: not to un-indent last line in References
               %nospread,     % flushend option: do not fill with whitespace to balance columns
               %hyphens,      % allow \url to hyphenate at "-" (hyphens)
               %xetex,        % use XeLaTeX to process the file
               %luatex,       % use LuaLaTeX to process the file
               ]{jacow}
%
% ONLY FOR \footnote in table/tabular
%
\usepackage{pdfpages,multirow,ragged2e} %
%
% CHANGE SEQUENCE OF GRAPHICS EXTENSION TO BE EMBEDDED
% ----------------------------------------------------
% test for XeTeX where the sequence is by default eps-> pdf, jpg, png, pdf, ...
%    and the JACoW template provides JACpic2v3.eps and JACpic2v3.jpg which
%    might generates errors, therefore PNG and JPG first
%
\makeatletter%
	\ifboolexpr{bool{xetex}}
	 {\renewcommand{\Gin@extensions}{.pdf,%
	                    .png,.jpg,.bmp,.pict,.tif,.psd,.mac,.sga,.tga,.gif,%
	                    .eps,.ps,%
	                    }}{}
\makeatother

% CHECK FOR XeTeX/LuaTeX BEFORE DEFINING AN INPUT ENCODING
% --------------------------------------------------------
%   utf8  is default for XeTeX/LuaTeX
%   utf8  in LaTeX only realises a small portion of codes
%
\ifboolexpr{bool{xetex} or bool{luatex}} % test for XeTeX/LuaTeX
 {}                                      % input encoding is utf8 by default
 {\usepackage[utf8]{inputenc}}           % switch to utf8

\usepackage[USenglish]{babel}

%
% if BibLaTeX is used
%
\ifboolexpr{bool{jacowbiblatex}}%
 {%
  \addbibresource{references.bib}
 }{}
\listfiles

%%
%%   Lengths for the spaces in the title
%%   \setlength\titleblockstartskip{..}  %before title, default 3pt
%%   \setlength\titleblockmiddleskip{..} %between title + author, default 1em
%%   \setlength\titleblockendskip{..}    %afterauthor, default 1em

\begin{document}

\title{Beam optics modelling through fringe fields during injection and extraction at the CERN Proton Synchrotron}

\author{E. P. Johnson\thanks{eliott.philippe.johnson@cern.ch}, M. A. Fraser, M. G. Atanasov, Y. Dutheil, E. Oponowicz,\\ CERN, 1211 Geneva 23, Switzerland}
	
\maketitle

%
\begin{abstract}
   Injection and extracted beam in the CERN Proton Synchrotron (PS) travels through it's combined function magnets. Although the region around the central orbit is well controlled, fringe field felt during main unit traversal must be modelled in order to improve the beam injection and extraction.
\end{abstract}


\section{Introduction}

The CERN PS is composed of a 100 combined function main unit magnets.

\subsection{AGS magnets}
The PS is a Alternating Gradient Synchrotron (AGS) that uses a strong focusing mechanism. It uses combined function magnet that have an half focusing and half defocussing part of opposite gradient of induction $\frac{dB}{dr}$.

\begin{figure}[!htb]
   \centering
   \includegraphics*[width=1.0\columnwidth]{magnet.png}
   \caption{Focusing and defocusing yokes}
   \label{fig:magnet}
\end{figure}

The main units of the PS are combined function magnet. The yoke provides both a dipolar and a quadrupolar field. Each magnet units consists in a combination of five focusing yokes followed by five defocusing yokes following the FDODFOFDODF pattern \cite{}

\begin{figure}[!htb]
   \centering
   \includegraphics*[width=1.0\columnwidth]{magnet_type.png}
   \caption{U, R, S, T}
   \label{fig:magnet_type}
\end{figure}

There are four types of magnets, U, R, S, T that have a different combination of focusing and defocusing half-magnet and the coil on the inside or outside of the ring.

Although the combined function magnet is a clever design when the beam is in orbit, extracting the beam through the main units causes stray fields effects.

\section{OPERA model}

In this reference a model of the PS magnet was developed \cite{anglada_reference_nodate}.

\begin{figure}[!htb]
   \centering
   \includegraphics*[width=1.0\columnwidth]{main_field.png}
   \caption{Dipole and Quadrupole field}
   \label{fig:main_field}
\end{figure}

\section{Injection BTP}

As the beam is injected via the BTP transfer line to the PS ring, it passes through the stray fields of the PR.BHT41 T-type main unit magnet. Once through, the beam is kicked by septum magnet PI.SMH42 to capture the reference orbit. Beam position can then be measured using a Secondary Emission Grid (SEM Grid) PI.BSG42.

\begin{figure}[!htb]
   \centering
   \includegraphics*[width=1.0\columnwidth]{kick_response_old_model.png}
   \caption{Kick response as a function of POPS}
   \label{fig:kick_response_old_model}
\end{figure}

\section{Extraction}

Extraction to East passes through two main units MU62 and MU63. As a result of the stray fields, instead of following a DFOFDODFOF lattice, the beam follows a DFOFDODD?D lattice. This succession in defocusing gradient increases the horizontal beam size.

\begin{figure*}[!htb]
   \centering
   \includegraphics*[width=\textwidth]{track_mu62.png}
   \caption{track mu62}
   \label{fig:track mu62}
\end{figure*}

\begin{figure}[!htb]
   \centering
   \includegraphics*[width=1.0\columnwidth]{gradient_track_mu62.png}
   \caption{gradient track mu62}
   \label{fig:gradient track mu62}
\end{figure}


\section{Magnetic shims}

Magnetic shims are installed at ejection 16 and 63 to homogenize the field by shielding the ejected beam from a non linear fringe field. In 16, the shims have different radial position for each of the five different shims.

Look at this shims in Fig. \ref{fig:shim}.

\begin{figure*}[!tbh]
    \centering
    \includegraphics*[width=\textwidth]{shim}
    \caption{Shim in the ejection 16}
    \label{fig:shim}
\end{figure*}


\section{CONCLUSION}

In conclusion,...

% REFERENCES
\printbibliography



\end{document}