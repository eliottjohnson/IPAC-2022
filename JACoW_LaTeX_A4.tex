\documentclass[a4paper,
               %boxit,        % check whether paper is inside correct margins
               %titlepage,    % separate title page
               %refpage       % separate references
               biblatex,     % biblatex is used
               keeplastbox,   % flushend option: not to un-indent last line in References
               %nospread,     % flushend option: do not fill with whitespace to balance columns
               %hyphens,      % allow \url to hyphenate at "-" (hyphens)
               %xetex,        % use XeLaTeX to process the file
               %luatex,       % use LuaLaTeX to process the file
               ]{jacow}
%
% ONLY FOR \footnote in table/tabular
%
\usepackage{pdfpages,multirow,ragged2e} %
%
% CHANGE SEQUENCE OF GRAPHICS EXTENSION TO BE EMBEDDED
% ----------------------------------------------------
% test for XeTeX where the sequence is by default eps-> pdf, jpg, png, pdf, ...
%    and the JACoW template provides JACpic2v3.eps and JACpic2v3.jpg which
%    might generates errors, therefore PNG and JPG first
%
\makeatletter%
	\ifboolexpr{bool{xetex}}
	 {\renewcommand{\Gin@extensions}{.pdf,%
	                    .png,.jpg,.bmp,.pict,.tif,.psd,.mac,.sga,.tga,.gif,%
	                    .eps,.ps,%
	                    }}{}
\makeatother

% CHECK FOR XeTeX/LuaTeX BEFORE DEFINING AN INPUT ENCODING
% --------------------------------------------------------
%   utf8  is default for XeTeX/LuaTeX
%   utf8  in LaTeX only realises a small portion of codes
%
\ifboolexpr{bool{xetex} or bool{luatex}} % test for XeTeX/LuaTeX
 {}                                      % input encoding is utf8 by default
 {\usepackage[utf8]{inputenc}}           % switch to utf8

\usepackage[USenglish]{babel}

%
% if BibLaTeX is used
%
\ifboolexpr{bool{jacowbiblatex}}%
 {%
  \addbibresource{references.bib}
 }{}
\listfiles

%%
%%   Lengths for the spaces in the title
%%   \setlength\titleblockstartskip{..}  %before title, default 3pt
%%   \setlength\titleblockmiddleskip{..} %between title + author, default 1em
%%   \setlength\titleblockendskip{..}    %afterauthor, default 1em

\begin{document}

\title{Beam optics modelling through fringe fields during injection and extraction at the CERN Proton Synchrotron}

\author{E. P. Johnson\thanks{eliott.philippe.johnson@cern.ch}, M. A. Fraser, M. G. Atanasov, Y. Dutheil, E. Oponowicz,\\ CERN, 1211 Geneva 23, Switzerland}
	
\maketitle

%
\begin{abstract}
   My abstract. Test.
\end{abstract}


\section{Introduction}

What is the PS

\subsection{AGS magnets}
Magnet are AGS.
\begin{Itemize}
    \item focusing
    \item defocusing
\end{Itemize}

Why are stray fields relevant in the PS ?

\section{OPERA model}

An model of the PS magnet was developed in this reference \cite{anglada_reference_nodate}.

\section{Injection BTP}

\begin{figure}[!htb]
   \centering
   \includegraphics*[width=1.0\columnwidth]{kick_response_old_model.png}
   \caption{Kick response as a function of POPS}
   \label{fig:kick_response_old_model}
\end{figure}

\section{Extraction}

\section{Shims}

Look at this shims in Fig. \ref{fig:shim}, it's so beautiful.

\begin{figure*}[!tbh]
    \centering
    \includegraphics*[width=\textwidth]{shim}

    \caption{Shim in the ejection 16}
    \label{fig:shim}
\end{figure*}


\section{CONCLUSION}

In conclusion,...

% REFERENCES
\printbibliography



\end{document}